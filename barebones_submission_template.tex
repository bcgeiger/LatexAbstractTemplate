\documentclass[11pt]{article}

% This file will be kept up-to-date at the following GitHub repository:
%
% https://github.com/automl-conf/LatexTemplate
%
% Please file any issues/bug reports, etc. you may have at:
%
% https://github.com/automl-conf/LatexTemplate/issues

\usepackage{microtype} % microtypography
\usepackage{booktabs}  % tables
\usepackage{url}  % urls
\usepackage{lipsum}

% AMS math
\usepackage{amsmath}
\usepackage{amsthm}

% With no package options, the submission will be anonymized, the supplemental
% material will be visible, and line numbers will be added to the manuscript.
%
% To compile a non-anonymized camera-ready version, add the [final] option:
%
% \usepackage[final]{automl}
%
% To compile for the workshop track, add the [workshop] option:
%
% \usepackage[workshop]{automl}
% \usepackage[final,workshop]{automl}
%
% To compile a single-blind submission, e.g. for the ABCD track, add the
% [revealauthors] option:
%
% \usepackage[revealauthors]{automl}
%
% To compile a non-anonymized pre-print version that can be uploaded to arXiv,
% add the [preprint] option:
%
% \usepackage[preprint]{automl}
%
% The [hidesupplement] option may be used to hide the supplementary material from any of the
% above commands, e.g.:
%
% \usepackage[hidesupplement,final,workshop]{automl}

\usepackage[workshop,final]{extabst}

% You may use any reference style as long as you are consistent throughout the
% document. As a default we suggest author--year citations; for bibtex and
% natbib you may use:

\usepackage{natbib}
\bibliographystyle{apalike}

% and for biber and biblatex you may use:

% \usepackage[%
%   backend=biber,
%   style=authoryear-comp,
%   sortcites=true,
%   natbib=true,
%   giveninits=true,
%   maxcitenames=2,
%   doi=false,
%   url=true,
%   isbn=false,
%   dashed=false
% ]{biblatex}
% \addbibresource{...}

\title{Example Submission for Green--AI\textsuperscript{2}--Energy 2025}

% The syntax for adding an author is
%
% \author[i]{\nameemail{author name}{author email}}
%
% where i is an affiliation counter. Authors may have
% multiple affiliations; e.g.:
%
% \author[1,2]{\nameemail{Anonymous}{anonymous@example.com}}

\author[1,$\ast$]{\nameemail{Author 1}{email1@example.com}}
\author[1,2,$\ast$]{\nameemail{Author 2}{email2@example.com}}
\author[2]{\nameemail{Author 3}{email3@example.com}}

\affil[$\ast$]{Equal contribution.}
\affil[1]{Institution 1}
\affil[2]{Institution 2}

% the list might continue:
% \author[2,3]{\nameemail{Author 2}{email2@example.com}}
% \author[3]{\nameemail{Author 3}{email3@example.com}}

% if you need to force a linebreak in the author list, prepend an \author entry
% with \\:

% \author[1,2,3]{\\\nameemail{Author 4}{email5@example.com}}

% Specify corresponding affiliations after authors, referring to counter used in
% \author:

% the list might continue:
% \affil[2]{Institution 2}
% \affil[3]{Institution 3}
%
% this system may also be used to add additional footnotes, e.g. to indicate
% equal contributions. Here is an example:
%
% \author[1,$\ast$]{\nameemail{Author 1}{email1@example.com}}
% \author[1,$\ast$]{\nameemail{Author 2}{email2@example.com}}
% \author[2]{\nameemail{Author 3}{email3@example.com}}
%
% \affil[1]{Institution 1}
% \affil[2]{Institution 2}
% \affil[$\ast$]{Equal contribution.}

% define PDF metadata, please fill in to aid in accessibility of the resulting PDF
\hypersetup{%
  pdfauthor={}, % will be reset to "Anonymous" unless the "final" package option is given
  pdftitle={},
  pdfsubject={},
  pdfkeywords={}
}

\begin{document}

\maketitle

% ==== The Abstract field can be left empty, as the full two pages count as an (extended) abstract.
% \begin{abstract}
% \end{abstract}

\section{Introduction}
\lipsum[1-2]



\section{Method and Experiments}
\begin{figure}
  \begin{subfigure}[t]{0.5\linewidth}
    \centering
    \framebox{Amazing figure!}
    \caption{Subfigure caption.}
    \label{example_figure_left}
  \end{subfigure}
  \begin{subfigure}[t]{0.5\linewidth}
    \centering
    \framebox{Another amazing figure!}
    \caption{Another subfigure caption.}
    \label{example_figure_right}
  \end{subfigure}
  \caption{An example figure with subfigures. \subref{example_figure_left}: an
    amazing figure. \subref{example_figure_right}: another amazing figure.}
  \label{example_figure}
\end{figure}

\begin{table}
  \caption{An example table using the \texttt{booktabs} package.}
  \label{example_table}
  \centering
  \begin{tabular}{lrr}
    \toprule
    & \multicolumn{2}{c}{metric} \\
    \cmidrule{2-3}
    method & accuracy & time \\
    \midrule
    baseline & 10 & 100 \\
    our method & \textbf{100} & \textbf{10} \\
    \bottomrule
  \end{tabular}
\end{table}

\lipsum[3-6]

\section{Discussion and Outlook}
\lipsum[7-8]

% ==== Formatting Instructions
% The page limit for the main paper is 9 pages; this includes the broader impact
% statement but not the submission checklist, references, or appendix.
% The broader impact statement and submission checklist are mandatory at both
% submission time and in the camera ready. References and supplemental materials
% are not limited in length. Accepted papers will be allowed to add an additional page
% of content to the main paper to react to reviewer feedback.
% This additional content may in fact be added during the rebuttal phase as authors
% interact with the reviewers to ensure acceptance decisions can be made regarding
% near-camera-ready work.


\begin{acknowledgements}
  \lipsum[9]
\end{acknowledgements}

% ==== Bibliography
% print bibliography -- for bibtex / natbib, use:

% \bibliography{...}

% and for biber / biblatex, use:

% \printbibliography

% supplemental material -- everything hereafter will be suppressed during
% submission time if the hidesupplement option is provided!

\end{document}
