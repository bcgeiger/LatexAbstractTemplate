\documentclass[11pt]{article}

% This file will be kept up-to-date at the following GitHub repository:
%
% https://github.com/automl-conf/LatexTemplate
%
% Please file any issues/bug reports, etc. you may have at:
%
% https://github.com/automl-conf/LatexTemplate/issues

\usepackage{microtype} % microtypography
\usepackage{booktabs}  % tables
\usepackage{url}  % urls
\usepackage{csquotes}

% AMS math
\usepackage{amsmath}
\usepackage{amsthm}


% With no package options, the submission will be anonymized, the supplemental
% material will be visible, and line numbers will be added to the manuscript.
%
% To compile a non-anonymized camera-ready version, add the [final] option:
%
% \usepackage[final]{automl}
%
% To compile for the workshop track, add the [workshop] option:
%
% \usepackage[workshop]{automl}
% \usepackage[final,workshop]{automl}
%
% To compile a single-blind submission, e.g. for the ABCD track, add the
% [revealauthors] option:
%
% \usepackage[revealauthors]{automl}
%
% To compile a non-anonymized pre-print version that can be uploaded to arXiv,
% add the [preprint] option:
%
% \usepackage[preprint]{automl}
%
% The [hidesupplement] option may be used to hide the supplementary material from any of the
% above commands, e.g.:
%
% \usepackage[hidesupplement,final,workshop]{automl}

\usepackage[final]{extabst}

% You may use any reference style as long as you are consistent throughout the
% document. As a default we suggest author--year citations; for bibtex and
% natbib you may use:

\usepackage{natbib}
\bibliographystyle{apalike}

% and for biber and biblatex you may use:

% \usepackage[%
%   backend=biber,
%   style=authoryear-comp,
%   sortcites=true,
%   natbib=true,
%   giveninits=true,
%   maxcitenames=2,
%   doi=false,
%   url=true,
%   isbn=false,
%   dashed=false
% ]{biblatex}
% \addbibresource{example.bib}

% example bibliography file
\usepackage{filecontents}
\begin{filecontents}{example.bib}
@book{example_book,
  author    = {Author, Anonymous},
  year      = {2000},
  title     = {The Definitive Resource},
  publisher = {Universal Press}
}
\end{filecontents}

\title{Documentation for the \texttt{extabst} Package}

% The syntax for adding an author is
%
% \author[i]{\nameemail{author name}{author email}}
%
% where i is an affiliation counter. Authors may have
% multiple affiliations; e.g.:
%
% \author[1,2]{\nameemail{Anonymous}{anonymous@example.com}}
%
% the list might continue:
% \author[2,3]{\nameemail{Author 2}{email2@example.com}}
% \author[3]{\nameemail{Author 3}{email3@example.com}}
%
% if you need to force a linebreak in the author list, prepend an \author entry
% with \\:
%
% \author[1,2,3]{\\\nameemail{Author 4}{email4@example.com}}
%
% Specify corresponding affiliations after authors, referring to counter used in
% \author:
%
% \affil[1]{Institution 1}
%
% the list might continue:
% \affil[2]{Institution 2}
% \affil[3]{Institution 3}
%
% this system may also be used to add additional footnotes, e.g. to indicate
% equal contributions. Here is an example:
%
% \author[1,$\ast$]{\nameemail{Author 1}{email1@example.com}}
% \author[1,$\ast$]{\nameemail{Author 2}{email2@example.com}}
% \author[2]{\nameemail{Author 3}{email3@example.com}}
%

\author[1,$\ast$]{\nameemail{Author 1}{email1@example.com}}
\author[1,2,$\ast$]{\nameemail{Author 2}{email2@example.com}}
\author[2]{\nameemail{Author 3}{email3@example.com}}

\affil[$\ast$]{Equal contribution.}
\affil[1]{Institution 1}
\affil[2]{Institution 2}

% define PDF metadata, aids in accessibility of the resulting PDF
\hypersetup{%
  pdfauthor={Modifier}, % will be reset to "Anonymous" unless the "final" package option is given
  pdftitle={Documentation for the extabst Package},
  pdfsubject={Documentation for extabst Package},
  pdfkeywords={extabst, LaTeX, style}
}

\begin{document}

\maketitle

\begin{abstract}
  Create with \verb|\begin{abstract} ... \end{abstract}|.
\end{abstract}

The \texttt{extabst} package provides a \LaTeX{} style for extended conference/workshop abstracts.
This document provides some notes regarding the package and tips for typesetting
manuscripts. The package and this document is maintained at the following GitHub
repository:
\begin{center}
  \url{https://github.com/bcgeiger/LatexAbstractTemplate}
\end{center}
which was forked from 
\begin{center}
  \url{https://github.com/automl-conf/LatexTemplate}
\end{center}

A barebones submission is also available as
\texttt{barebones\_submission\_template.tex} in the same repository.

\section{Package Options}

With no options, the \texttt{extabst} package prepares an anonymized manuscript
ready for submission. Several options are supported changing
this behavior:
\begin{itemize}
\item \texttt{revealauthors} -- produces a non-anonymized manuscript for
  single-blind submission, e.g. to the ABCD track
\item \texttt{workshop} -- produces a manuscript for submission and/or
  publication in the workshop track
\item \texttt{preprint} -- produces a non-anonymized manuscript for
  preliminary (i.e., pre-acceptance) distribution, e.g. in arXiv
\item \texttt{final} -- produces a non-anonymized manuscript for distribution
  and/or publication in the proceedings
\item \texttt{hidesupplement} -- hides supplementary material (following
  \verb|\appendix|); for example, for submitting or distributing the main paper
  without supplement
\end{itemize}
These options may be combined. For example, \texttt{final} or \texttt{workshop}
may be used together to create a camera-ready manuscript for the workshop track,
and \texttt{hidesupplement} can be used in combination with any other flags to
hide the supplemental material.

\section{Supplemental Material}

Please provide supplemental material in the main document. You may begin the
supplemental material using \verb|\appendix|. Any content following this command
will be suppressed in the final output if the \texttt{hidesupplement} option is
given.

\section{Note Regarding Line Numbering at Submission Time}

To ensure that line numbering works correctly with display math mode, please do
\emph{not} use \TeX{} primitives such as \verb|$$| and \texttt{eqnarray}.  (Using
these is not good practice anyway.)%
%
\footnote{\url{https://tex.stackexchange.com/questions/196/eqnarray-vs-align}}%
\footnote{\url{https://tex.stackexchange.com/questions/503/why-is-preferable-to}}
%
Please use \LaTeX{} equivalents such as \verb|\[ ... \]| (or
\verb|\begin{equation} ... \end{equation}|) and the \texttt{align} environment
from the \texttt{amsmath} package.%
%
\footnote{\url{http://tug.ctan.org/info/short-math-guide/short-math-guide.pdf}}

\section{References}

Authors may use any citation style as long as it is consistent throughout the
document. By default we propose author--year citations. Code is provided in the
preamble to achieve such citations using either \texttt{natbib/bibtex} or the
more modern \texttt{biblatex/biber}.

You may create a parenthetical reference with \verb|\citep|, such as appears at
the end of this sentence \citep{example_book}.  You may create a textual
reference using \verb|\citet|, as \citet{example_book} also demonstrated.

\section{Tables}

We recommend the \texttt{booktabs} package for creating tables, as demonstrated
in Table \ref{example_table}.

Table captions should appear \emph{above} tables.

\begin{table}
  \caption{An example table using the \texttt{booktabs} package.}
  \label{example_table}
  \centering
  \begin{tabular}{lrr}
    \toprule
    & \multicolumn{2}{c}{metric} \\
    \cmidrule{2-3}
    method & accuracy & time \\
    \midrule
    baseline & 10 & 100 \\
    our method & \textbf{100} & \textbf{10} \\
    \bottomrule
  \end{tabular}
\end{table}

\section{Figures and Subfigures}

The \texttt{automl} style loads the \texttt{subcaption} package, which may be
used to create and caption subfigures. Please note that this is
\emph{incompatible} with the (obsolete and deprecated) \texttt{subfigure}
package. A figure with subfigures is demonstrated in Figure
\ref{example_figure}.

Figure captions should appear \emph{below} figures.

Please ensure that all text appearing in figures (axis labels, legends, etc.)
is legible.

\begin{figure}
  \begin{subfigure}[t]{0.5\linewidth}
    \centering
    \framebox{Amazing figure!}
    \caption{Subfigure caption.}
    \label{example_figure_left}
  \end{subfigure}
  \begin{subfigure}[t]{0.5\linewidth}
    \centering
    \framebox{Another amazing figure!}
    \caption{Another subfigure caption.}
    \label{example_figure_right}
  \end{subfigure}
  \caption{An example figure with subfigures. \subref{example_figure_left}: an
    amazing figure. \subref{example_figure_right}: another amazing figure.}
  \label{example_figure}
\end{figure}

\section{Pseudocode}
\label{sec:code}

To add pseudocode, you may make use of any package you see fit -- the
\texttt{automl} package should be compatible with any of them. In particular,
you may want to check out the \texttt{algorithm2e}%
%
\footnote{\url{https://ctan.org/pkg/algorithm2e}}
%
and/or the \texttt{algorithmicx}%
%
\footnote{\url{https://ctan.org/pkg/algorithmicx}}
%
packages, both of which can produce nicely typeset pseudocode. You may also wish
to load the \texttt{algorithm}%
%
\footnote{\url{https://ctan.org/pkg/algorithms}}
%
package, which creates an \texttt{algorithm} floating environment you can access
with \verb|\begin{algorithm} ... \end{algorithm}|. This environment supports
\verb|\caption{}|, \verb|\label{}| and \verb|\ref{}|, etc.

\section{Adding Acknowledgments}

You may add acknowledgments of funding, etc.\ using the \texttt{acknowledgments}
environment. Acknowledgments will be automatically commented out in anonymized
manuscripts (that is, if the \texttt{final} or \texttt{preprint} options are not
given). An example is given below in the source code for this document; it will
be hidden in the \textsc{pdf} unless the \texttt{final} or
\texttt{finalworkshop} option is given.

\section{Required Material}

All submissions must include a broader impact statement and a
\textbf{Reproducibility Checklist} in their manuscripts, both at submission and
camera-ready time.  The discussion of limitations and broader impact is part of
the 9 pages allocated to the main paper (there is no page limitation for
references and appendices), while the reproducibility checklist is not.

\section{Broader Impact Statement}

The 9 pages allocated for the main paper must include a broader impact statement
regarding the approach, datasets and applications proposed/used in your
paper. It should reflect on the environmental, ethical and societal implications
of your work. The statement should require at most one page and must be included
both at submission and camera-ready time.

If authors have reflected on their work and determined that there are no likely
negative broader impacts, they may use the following statement:

\begin{displayquote}
  After careful reflection, the authors have determined that this work presents no notable
  negative impacts to society or the environment.
\end{displayquote}

A standalone section for the broader impact statement is included in the
template by default, but you can also place this discussion anywhere else in the
paper, e.g., in the introduction or a future work section.

The Centre for the Governance of AI has written an excellent guide for writing
good broader impact statements (for the NeurIPS conference) that may be a useful
resource for AutoML authors.%
%
\footnote{\url{https://medium.com/@GovAI/a-guide-to-writing-the-neurips-impact-statement-4293b723f832}}

\begin{acknowledgements}
  The authors have many people to thank! This material will be automatically
  hidden for submissions, e.g., if the \texttt{final} or \texttt{preprint}
  options are not given.
\end{acknowledgements}

% print bibliography -- for bibtex / natbib, use:

\bibliography{example}

% and for biber / biblatex, use:

% \printbibliography

% supplemental material -- everything hereafter will be suppressed during
% submission time if the hidesupplement option is provided!

\newpage
\appendix
\section{Proof of Theorem 1}

This material will be hidden if the \texttt{hidesupplement} option is provided.

\end{document}
